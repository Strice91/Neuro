\documentclass[conference]{IEEEtran}

% Packages
\usepackage[utf8]{inputenc}
\usepackage[english]{babel}
\usepackage{amsmath}
\usepackage{amsfonts}
\usepackage{amssymb}
\usepackage{amsthm}
\usepackage{pdfpages}
\usepackage{graphicx}
\usepackage{epstopdf}
\usepackage{listings}
\usepackage{cite}
\usepackage{enumerate}
\usepackage{scientific}
\usepackage[colorlinks=false]{hyperref}
\usepackage{bookmark}

\usepackage[]{mcode}	%Matlab Code
\usepackage{tikz,pgfplots}	%Tikz

%\usepgfplotslibrary{external} 
%\tikzexternalize
%\tikzsetexternalprefix{ext/}

% Bookmark Setup
\bookmarksetup{numbered}

% PDF Setup
\hypersetup{pdftitle={Homework 6}, pdfsubject={Documentation of 6th Homework}, pdfauthor={Stefan Röhrl}, pdfkeywords={Neuroprothetik Exercise}, pdfcreator={LaTeX}, hidelinks}


\begin{document}
%
% cite all references
%\nocite{*}
%
% paper title
% can use linebreaks \\ within to get better formatting as desired
\title{Homework 6\\ Electric Stimulation}

\author{\IEEEauthorblockN{Stefan Röhrl}
\IEEEauthorblockA{Technische Universität München, Arcisstraße 21, Munich, Germany\\
Email: stefan.roehrl@tum.de}}

% use for special paper notices
%\IEEEspecialpapernotice{(Invited Paper)}

% make the title area
\maketitle

\IEEEpeerreviewmaketitle

\section{Calculate the Potential Field}

Das Potential auf einer der ebenen Fläche wird wie folgt berechnet:
\begin{lstlisting}
% Koordinaten in der X,Y-Ebene
[X,Y] =  meshgrid(-dim(1)/2:0.1:dim(1)/2,...
				  -dim(2)/2:0.1:dim(2)/2);
% Radialer Abstand
r = sqrt(X.^2 + Y.^2 + z^2);
% Potential
Phi = (rho_medium * I) ./ (4 * pi .* r);
\end{lstlisting}

\begin{enumerate}
\item Der Potenzialverlauf auf einer $50\mu m$ x $50\mu m$ Ebene über der eine Elektrode in $10\mu m$ Abstand angebracht ist in Abbildung \ref{fig:potField} dargestellt. Der Anregende Strom hat eine Größe von 1mA.

\begin{figure}[h]
	\centering
	\includegraphics[width=0.5\textwidth]{img/potField.png}
	\caption{Potenzialfrei auf der Ebene}
	\label{fig:potField}
\end{figure}

\item Die folgenden drei Graphen zeigen den Potenzialverlauf (\ref{fig:Phi1}), das elektrische Feld (\ref{fig:E1}) und die Aktivierungsfunktion (\ref{fig:A1}) bei einem Elektrodenstrom von 1mA.\\

\begin{figure}[h!]
  	\centering
    \scalebox{.6}{% This file was created by matlab2tikz.
% Minimal pgfplots version: 1.3
%
%The latest updates can be retrieved from
%  http://www.mathworks.com/matlabcentral/fileexchange/22022-matlab2tikz
%where you can also make suggestions and rate matlab2tikz.
%
\begin{tikzpicture}

\begin{axis}[%
width=4.520833in,
height=3.565625in,
at={(0.758333in,0.48125in)},
scale only axis,
separate axis lines,
every outer x axis line/.append style={black},
every x tick label/.append style={font=\color{black}},
xmin=1,
xmax=501,
xtick={1,51,101,151,201,251,301,351,401,451,501},
xticklabels={{-25},{-20},{-15},{-10},{-5},{0},{5},{10},{15},{20},{25}},
xlabel={$\text{x distance in }\mu\text{m}$},
every outer y axis line/.append style={black},
every y tick label/.append style={font=\color{black}},
ymin=8,
ymax=24,
ylabel={$\phi\text{ in V}$},
title={$\phi\text{ with I = 1mA}$}
]
\addplot [color=blue,solid,forget plot]
  table[row sep=crcr]{%
1	8.86629929399968\\
2	8.8969700084153\\
3	8.92783637312795\\
4	8.95890006992601\\
5	8.99016279678701\\
6	9.02162626800457\\
7	9.0532922143142\\
8	9.08516238301814\\
9	9.11723853810915\\
10	9.14952246039295\\
11	9.18201594760935\\
12	9.2147208145518\\
13	9.24763889318544\\
14	9.28077203276323\\
15	9.31412209994021\\
16	9.34769097888575\\
17	9.38148057139342\\
18	9.41549279698855\\
19	9.4497295930332\\
20	9.48419291482831\\
21	9.51888473571293\\
22	9.55380704716026\\
23	9.58896185887038\\
24	9.62435119885929\\
25	9.65997711354423\\
26	9.69584166782492\\
27	9.73194694516044\\
28	9.76829504764166\\
29	9.80488809605876\\
30	9.84172822996365\\
31	9.87881760772705\\
32	9.91615840658981\\
33	9.95375282270821\\
34	9.99160307119295\\
35	10.0297113861414\\
36	10.0680800206629\\
37	10.1067112468965\\
38	10.1456073560207\\
39	10.1847706582558\\
40	10.2242034828564\\
41	10.2639081780958\\
42	10.3038871112406\\
43	10.3441426685148\\
44	10.3846772550542\\
45	10.4254932948493\\
46	10.4665932306765\\
47	10.5079795240181\\
48	10.5496546549681\\
49	10.5916211221265\\
50	10.633881442478\\
51	10.6764381512577\\
52	10.7192938018002\\
53	10.762450965374\\
54	10.8059122309984\\
55	10.8496802052432\\
56	10.8937575120104\\
57	10.9381467922972\\
58	10.9828507039382\\
59	11.0278719213278\\
60	11.0732131351206\\
61	11.1188770519094\\
62	11.1648663938792\\
63	11.2111838984375\\
64	11.257832317818\\
65	11.3048144186584\\
66	11.3521329815496\\
67	11.3997908005568\\
68	11.4477906827095\\
69	11.4961354474605\\
70	11.5448279261121\\
71	11.5938709612075\\
72	11.6432674058876\\
73	11.6930201232103\\
74	11.743131985431\\
75	11.7936058732433\\
76	11.8444446749783\\
77	11.8956512857599\\
78	11.9472286066156\\
79	11.9991795435399\\
80	12.0515070065092\\
81	12.1042139084464\\
82	12.1573031641322\\
83	12.2107776890623\\
84	12.2646403982476\\
85	12.3188942049555\\
86	12.3735420193902\\
87	12.4285867473089\\
88	12.4840312885732\\
89	12.5398785356309\\
90	12.5961313719278\\
91	12.6527926702456\\
92	12.7098652909639\\
93	12.7673520802425\\
94	12.8252558681227\\
95	12.8835794665432\\
96	12.9423256672684\\
97	13.0014972397258\\
98	13.061096928749\\
99	13.1211274522235\\
100	13.1815914986317\\
101	13.2424917244934\\
102	13.3038307516985\\
103	13.3656111647286\\
104	13.4278355077628\\
105	13.4905062816651\\
106	13.5536259408487\\
107	13.6171968900131\\
108	13.6812214807503\\
109	13.7457020080167\\
110	13.8106407064644\\
111	13.8760397466298\\
112	13.9419012309738\\
113	14.0082271897692\\
114	14.0750195768319\\
115	14.1422802650897\\
116	14.2100110419851\\
117	14.2782136047073\\
118	14.3468895552481\\
119	14.4160403952782\\
120	14.4856675208368\\
121	14.5557722168323\\
122	14.6263556513477\\
123	14.6974188697452\\
124	14.7689627885681\\
125	14.8409881892309\\
126	14.913495711497\\
127	14.9864858467361\\
128	15.0599589309589\\
129	15.1339151376226\\
130	15.2083544702045\\
131	15.2832767545379\\
132	15.3586816309074\\
133	15.434568545898\\
134	15.5109367439959\\
135	15.5877852589363\\
136	15.6651129047951\\
137	15.7429182668212\\
138	15.8211996920078\\
139	15.8999552793983\\
140	15.9791828701272\\
141	16.0588800371924\\
142	16.1390440749589\\
143	16.2196719883934\\
144	16.3007604820292\\
145	16.3823059486627\\
146	16.4643044577823\\
147	16.5467517437324\\
148	16.6296431936146\\
149	16.7129738349299\\
150	16.7967383229679\\
151	16.8809309279457\\
152	16.9655455219068\\
153	17.050575565384\\
154	17.1360140938388\\
155	17.2218537038844\\
156	17.3080865393063\\
157	17.3947042768919\\
158	17.4816981120845\\
159	17.5690587444772\\
160	17.6567763631654\\
161	17.744840631976\\
162	17.833240674596\\
163	17.9219650596237\\
164	18.0110017855666\\
165	18.1003382658155\\
166	18.1899613136226\\
167	18.2798571271167\\
168	18.370011274389\\
169	18.4604086786875\\
170	18.5510336037574\\
171	18.6418696393712\\
172	18.7328996870913\\
173	18.8241059463135\\
174	18.9154699006402\\
175	19.0069723046371\\
176	19.0985931710274\\
177	19.1903117583826\\
178	19.2821065593705\\
179	19.3739552896247\\
180	19.465834877301\\
181	19.5577214533908\\
182	19.649590342863\\
183	19.7414160567084\\
184	19.8331722849637\\
185	19.9248318907938\\
186	20.0163669057136\\
187	20.1077485260321\\
188	20.1989471106031\\
189	20.2899321799699\\
190	20.3806724169889\\
191	20.4711356690224\\
192	20.5612889517878\\
193	20.6510984549515\\
194	20.7405295495568\\
195	20.8295467973721\\
196	20.9181139622468\\
197	21.0061940235573\\
198	21.0937491918274\\
199	21.1807409265998\\
200	21.2671299566362\\
201	21.3528763025153\\
202	21.4379393016965\\
203	21.5222776361092\\
204	21.6058493623217\\
205	21.6886119443383\\
206	21.7705222890629\\
207	21.8515367844602\\
208	21.9316113404367\\
209	22.0107014324528\\
210	22.0887621478666\\
211	22.1657482349992\\
212	22.2416141548987\\
213	22.3163141357665\\
214	22.3898022299986\\
215	22.4620323737772\\
216	22.5329584491365\\
217	22.6025343484108\\
218	22.6707140409568\\
219	22.7374516420283\\
220	22.8027014836653\\
221	22.8664181874441\\
222	22.9285567389191\\
223	22.989072563573\\
224	23.0479216040761\\
225	23.1050603986414\\
226	23.1604461602489\\
227	23.2140368564986\\
228	23.2657912898409\\
229	23.3156691779195\\
230	23.3636312337557\\
231	23.4096392454901\\
232	23.4536561553939\\
233	23.4956461378539\\
234	23.5355746760321\\
235	23.5734086368993\\
236	23.6091163443391\\
237	23.6426676500222\\
238	23.6740340017532\\
239	23.7031885089971\\
240	23.7301060053\\
241	23.7547631073285\\
242	23.7771382702615\\
243	23.7972118392823\\
244	23.8149660969353\\
245	23.830385306124\\
246	23.8434557485501\\
247	23.85416575841\\
248	23.8625057511884\\
249	23.8684682474087\\
250	23.8720478912283\\
251	23.8732414637843\\
252	23.8720478912283\\
253	23.8684682474087\\
254	23.8625057511884\\
255	23.85416575841\\
256	23.8434557485501\\
257	23.830385306124\\
258	23.8149660969353\\
259	23.7972118392823\\
260	23.7771382702615\\
261	23.7547631073285\\
262	23.7301060053\\
263	23.7031885089971\\
264	23.6740340017532\\
265	23.6426676500222\\
266	23.6091163443391\\
267	23.5734086368993\\
268	23.5355746760321\\
269	23.4956461378539\\
270	23.4536561553939\\
271	23.4096392454901\\
272	23.3636312337557\\
273	23.3156691779195\\
274	23.2657912898409\\
275	23.2140368564986\\
276	23.1604461602489\\
277	23.1050603986414\\
278	23.0479216040761\\
279	22.989072563573\\
280	22.9285567389191\\
281	22.8664181874441\\
282	22.8027014836653\\
283	22.7374516420283\\
284	22.6707140409568\\
285	22.6025343484108\\
286	22.5329584491365\\
287	22.4620323737772\\
288	22.3898022299986\\
289	22.3163141357665\\
290	22.2416141548987\\
291	22.1657482349992\\
292	22.0887621478666\\
293	22.0107014324528\\
294	21.9316113404367\\
295	21.8515367844602\\
296	21.7705222890629\\
297	21.6886119443383\\
298	21.6058493623217\\
299	21.5222776361092\\
300	21.4379393016965\\
301	21.3528763025153\\
302	21.2671299566362\\
303	21.1807409265998\\
304	21.0937491918274\\
305	21.0061940235573\\
306	20.9181139622468\\
307	20.8295467973721\\
308	20.7405295495568\\
309	20.6510984549515\\
310	20.5612889517878\\
311	20.4711356690224\\
312	20.3806724169889\\
313	20.2899321799699\\
314	20.1989471106031\\
315	20.1077485260321\\
316	20.0163669057136\\
317	19.9248318907938\\
318	19.8331722849637\\
319	19.7414160567084\\
320	19.649590342863\\
321	19.5577214533908\\
322	19.465834877301\\
323	19.3739552896247\\
324	19.2821065593705\\
325	19.1903117583826\\
326	19.0985931710274\\
327	19.0069723046371\\
328	18.9154699006402\\
329	18.8241059463135\\
330	18.7328996870913\\
331	18.6418696393712\\
332	18.5510336037574\\
333	18.4604086786875\\
334	18.370011274389\\
335	18.2798571271167\\
336	18.1899613136226\\
337	18.1003382658155\\
338	18.0110017855666\\
339	17.9219650596237\\
340	17.833240674596\\
341	17.744840631976\\
342	17.6567763631654\\
343	17.5690587444772\\
344	17.4816981120845\\
345	17.3947042768919\\
346	17.3080865393063\\
347	17.2218537038844\\
348	17.1360140938388\\
349	17.050575565384\\
350	16.9655455219068\\
351	16.8809309279457\\
352	16.7967383229679\\
353	16.7129738349299\\
354	16.6296431936146\\
355	16.5467517437324\\
356	16.4643044577823\\
357	16.3823059486627\\
358	16.3007604820292\\
359	16.2196719883934\\
360	16.1390440749589\\
361	16.0588800371924\\
362	15.9791828701272\\
363	15.8999552793983\\
364	15.8211996920078\\
365	15.7429182668212\\
366	15.6651129047951\\
367	15.5877852589363\\
368	15.5109367439959\\
369	15.434568545898\\
370	15.3586816309074\\
371	15.2832767545379\\
372	15.2083544702045\\
373	15.1339151376226\\
374	15.0599589309589\\
375	14.9864858467361\\
376	14.913495711497\\
377	14.8409881892309\\
378	14.7689627885681\\
379	14.6974188697452\\
380	14.6263556513477\\
381	14.5557722168323\\
382	14.4856675208368\\
383	14.4160403952782\\
384	14.3468895552481\\
385	14.2782136047073\\
386	14.2100110419851\\
387	14.1422802650897\\
388	14.0750195768319\\
389	14.0082271897692\\
390	13.9419012309738\\
391	13.8760397466298\\
392	13.8106407064644\\
393	13.7457020080167\\
394	13.6812214807503\\
395	13.6171968900131\\
396	13.5536259408487\\
397	13.4905062816651\\
398	13.4278355077628\\
399	13.3656111647286\\
400	13.3038307516985\\
401	13.2424917244934\\
402	13.1815914986317\\
403	13.1211274522235\\
404	13.061096928749\\
405	13.0014972397258\\
406	12.9423256672684\\
407	12.8835794665432\\
408	12.8252558681227\\
409	12.7673520802425\\
410	12.7098652909639\\
411	12.6527926702456\\
412	12.5961313719278\\
413	12.5398785356309\\
414	12.4840312885732\\
415	12.4285867473089\\
416	12.3735420193902\\
417	12.3188942049555\\
418	12.2646403982476\\
419	12.2107776890623\\
420	12.1573031641322\\
421	12.1042139084464\\
422	12.0515070065092\\
423	11.9991795435399\\
424	11.9472286066156\\
425	11.8956512857599\\
426	11.8444446749783\\
427	11.7936058732433\\
428	11.743131985431\\
429	11.6930201232103\\
430	11.6432674058876\\
431	11.5938709612075\\
432	11.5448279261121\\
433	11.4961354474605\\
434	11.4477906827095\\
435	11.3997908005568\\
436	11.3521329815496\\
437	11.3048144186584\\
438	11.257832317818\\
439	11.2111838984375\\
440	11.1648663938792\\
441	11.1188770519094\\
442	11.0732131351206\\
443	11.0278719213278\\
444	10.9828507039382\\
445	10.9381467922972\\
446	10.8937575120104\\
447	10.8496802052432\\
448	10.8059122309984\\
449	10.762450965374\\
450	10.7192938018002\\
451	10.6764381512577\\
452	10.633881442478\\
453	10.5916211221265\\
454	10.5496546549681\\
455	10.5079795240181\\
456	10.4665932306765\\
457	10.4254932948493\\
458	10.3846772550542\\
459	10.3441426685148\\
460	10.3038871112406\\
461	10.2639081780958\\
462	10.2242034828564\\
463	10.1847706582558\\
464	10.1456073560207\\
465	10.1067112468965\\
466	10.0680800206629\\
467	10.0297113861414\\
468	9.99160307119295\\
469	9.95375282270821\\
470	9.91615840658981\\
471	9.87881760772705\\
472	9.84172822996365\\
473	9.80488809605876\\
474	9.76829504764166\\
475	9.73194694516044\\
476	9.69584166782492\\
477	9.65997711354423\\
478	9.62435119885929\\
479	9.58896185887038\\
480	9.55380704716026\\
481	9.51888473571293\\
482	9.48419291482831\\
483	9.4497295930332\\
484	9.41549279698855\\
485	9.38148057139342\\
486	9.34769097888575\\
487	9.31412209994021\\
488	9.28077203276323\\
489	9.24763889318544\\
490	9.2147208145518\\
491	9.18201594760935\\
492	9.14952246039295\\
493	9.11723853810915\\
494	9.08516238301814\\
495	9.0532922143142\\
496	9.02162626800457\\
497	8.99016279678701\\
498	8.95890006992601\\
499	8.92783637312795\\
500	8.8969700084153\\
501	8.86629929399968\\
};
\end{axis}
\end{tikzpicture}%}
    \caption{Potenzialverlauf im Axon bei I = 1mA}
    \label{fig:Phi1}
\end{figure}

\begin{figure}[h!]
  	\centering
    \scalebox{.6}{% This file was created by matlab2tikz.
% Minimal pgfplots version: 1.3
%
%The latest updates can be retrieved from
%  http://www.mathworks.com/matlabcentral/fileexchange/22022-matlab2tikz
%where you can also make suggestions and rate matlab2tikz.
%
\begin{tikzpicture}

\begin{axis}[%
width=4.520833in,
height=3.565625in,
at={(0.758333in,0.48125in)},
scale only axis,
separate axis lines,
every outer x axis line/.append style={black},
every x tick label/.append style={font=\color{black}},
xmin=1,
xmax=501,
xtick={1,51,101,151,201,251,301,351,401,451,501},
xticklabels={{-25},{-20},{-15},{-10},{-5},{0},{5},{10},{15},{20},{25}},
xlabel={$\text{x distance in }\mu\text{m}$},
every outer y axis line/.append style={black},
every y tick label/.append style={font=\color{black}},
ymin=-1,
ymax=1,
ylabel={$\text{E in V/}\mu\text{m}$},
title={E with I = 1mA}
]
\addplot [color=blue,solid,forget plot]
  table[row sep=crcr]{%
1	-0.306707144156118\\
2	-0.308663647126579\\
3	-0.310636967980518\\
4	-0.312627268610068\\
5	-0.314634712175614\\
6	-0.316659463096229\\
7	-0.318701687039393\\
8	-0.32076155091012\\
9	-0.322839222838027\\
10	-0.32493487216394\\
11	-0.327048669424581\\
12	-0.329180786336387\\
13	-0.331331395777834\\
14	-0.333500671769862\\
15	-0.335688789455411\\
16	-0.337895925076648\\
17	-0.340122255951325\\
18	-0.342367960446506\\
19	-0.344633217951085\\
20	-0.346918208846176\\
21	-0.349223114473336\\
22	-0.351548117101199\\
23	-0.353893399889067\\
24	-0.35625914684946\\
25	-0.358645542806855\\
26	-0.361052773355208\\
27	-0.363481024812238\\
28	-0.36593048417096\\
29	-0.368401339048887\\
30	-0.370893777634009\\
31	-0.373407988627594\\
32	-0.37594416118397\\
33	-0.378502484847481\\
34	-0.381083149484756\\
35	-0.383686345215093\\
36	-0.38631226233532\\
37	-0.388961091242361\\
38	-0.39163302235071\\
39	-0.394328246006062\\
40	-0.39704695239454\\
41	-0.399789331447735\\
42	-0.402555572741985\\
43	-0.405345865394047\\
44	-0.408160397950539\\
45	-0.410999358272743\\
46	-0.413862933415476\\
47	-0.416751309500434\\
48	-0.419664671583373\\
49	-0.422603203515255\\
50	-0.425567087796601\\
51	-0.428556505425508\\
52	-0.431571635738397\\
53	-0.434612656243658\\
54	-0.437679742447603\\
55	-0.440773067672762\\
56	-0.443892802867989\\
57	-0.447039116409851\\
58	-0.450212173895892\\
59	-0.453412137928133\\
60	-0.456639167887527\\
61	-0.459893419698432\\
62	-0.463175045582975\\
63	-0.466484193805172\\
64	-0.469821008403297\\
65	-0.473185628912027\\
66	-0.47657819007199\\
67	-0.479998821527143\\
68	-0.483447647510538\\
69	-0.486924786515335\\
70	-0.490430350953872\\
71	-0.493964446801574\\
72	-0.497527173227006\\
73	-0.501118622206533\\
74	-0.50473887812311\\
75	-0.508388017350008\\
76	-0.512066107816427\\
77	-0.515773208556922\\
78	-0.519509369242517\\
79	-0.52327462969318\\
80	-0.527069019371833\\
81	-0.530892556857978\\
82	-0.534745249301025\\
83	-0.538627091853066\\
84	-0.542538067079441\\
85	-0.546478144346541\\
86	-0.550447279187516\\
87	-0.554445412642739\\
88	-0.558472470576685\\
89	-0.562528362968973\\
90	-0.566612983178718\\
91	-0.570726207182872\\
92	-0.57486789278606\\
93	-0.579037878801802\\
94	-0.58323598420472\\
95	-0.587462007251975\\
96	-0.591715724573767\\
97	-0.595996890231945\\
98	-0.600305234745502\\
99	-0.604640464082369\\
100	-0.609002258616833\\
101	-0.613390272051006\\
102	-0.617804130300428\\
103	-0.6222434303419\\
104	-0.626707739023669\\
105	-0.631196591836165\\
106	-0.63570949164319\\
107	-0.640245907372314\\
108	-0.644805272663849\\
109	-0.649386984477172\\
110	-0.653990401654259\\
111	-0.658614843439498\\
112	-0.663259587954386\\
113	-0.667923870627352\\
114	-0.672606882577984\\
115	-0.67730776895381\\
116	-0.682025627221385\\
117	-0.686759505408663\\
118	-0.69150840030062\\
119	-0.696271255585952\\
120	-0.701046959955587\\
121	-0.705834345153065\\
122	-0.710632183975815\\
123	-0.715439188228579\\
124	-0.720254006628203\\
125	-0.725075222660738\\
126	-0.729901352391416\\
127	-0.734730842227727\\
128	-0.739562066636967\\
129	-0.744393325818713\\
130	-0.749222843334518\\
131	-0.754048763694684\\
132	-0.758869149905763\\
133	-0.763681980979314\\
134	-0.768485149404494\\
135	-0.773276458587677\\
136	-0.778053620261314\\
137	-0.782814251865371\\
138	-0.787555873905088\\
139	-0.792275907289213\\
140	-0.796971670651878\\
141	-0.801640377665152\\
142	-0.806279134344869\\
143	-0.810884936357859\\
144	-0.81545466633461\\
145	-0.819985091196251\\
146	-0.824472859501348\\
147	-0.82891449882144\\
148	-0.833306413153778\\
149	-0.8376448803795\\
150	-0.841926049778579\\
151	-0.846145939610352\\
152	-0.850300434772322\\
153	-0.854385284547767\\
154	-0.858396100456034\\
155	-0.862328354218995\\
156	-0.866177375856125\\
157	-0.869938351925796\\
158	-0.87360632392766\\
159	-0.877176186882025\\
160	-0.880642688105553\\
161	-0.884000426200195\\
162	-0.887243850276391\\
163	-0.890367259429148\\
164	-0.893364802489351\\
165	-0.896230478071338\\
166	-0.898958134940528\\
167	-0.901541472723153\\
168	-0.90397404298475\\
169	-0.90624925069946\\
170	-0.908360356137976\\
171	-0.910300477200998\\
172	-0.912062592221332\\
173	-0.913639543267308\\
174	-0.915024039969445\\
175	-0.916208663903006\\
176	-0.91718587355146\\
177	-0.917948009879126\\
178	-0.918487302542061\\
179	-0.91879587676285\\
180	-0.918865760898129\\
181	-0.918688894721882\\
182	-0.918257138453811\\
183	-0.917562282552922\\
184	-0.916596058301238\\
185	-0.915350149198311\\
186	-0.913816203184545\\
187	-0.911985845710355\\
188	-0.909850693667522\\
189	-0.907402370189878\\
190	-0.904632520335689\\
191	-0.901532827654208\\
192	-0.898095031636927\\
193	-0.894310946052315\\
194	-0.890172478153417\\
195	-0.885671648746609\\
196	-0.88080061310535\\
197	-0.875551682700753\\
198	-0.86991734772436\\
199	-0.863890300363934\\
200	-0.857463458791052\\
201	-0.850629991812362\\
202	-0.843383344126423\\
203	-0.83571726212476\\
204	-0.827625820166595\\
205	-0.819103447246157\\
206	-0.810144953973015\\
207	-0.800745559765019\\
208	-0.790900920160738\\
209	-0.780607154137627\\
210	-0.769860871326351\\
211	-0.7586591989946\\
212	-0.746999808678481\\
213	-0.734880942321112\\
214	-0.722301437785546\\
215	-0.709260753593099\\
216	-0.695758992743407\\
217	-0.681796925459963\\
218	-0.667376010714733\\
219	-0.652498416369944\\
220	-0.637167037787734\\
221	-0.621385514749981\\
222	-0.605158246539297\\
223	-0.588490405031195\\
224	-0.57138794565315\\
225	-0.553857616074644\\
226	-0.535906962497315\\
227	-0.517544333422322\\
228	-0.498778880785942\\
229	-0.479620558361908\\
230	-0.460080117344503\\
231	-0.440169099038421\\
232	-0.419899824599383\\
233	-0.399285381781773\\
234	-0.378339608672036\\
235	-0.357077074397942\\
236	-0.335513056831296\\
237	-0.31366351731009\\
238	-0.291545072438595\\
239	-0.269174963029286\\
240	-0.246571020285558\\
241	-0.223751629329243\\
242	-0.200735690208518\\
243	-0.177542576530065\\
244	-0.154192091887033\\
245	-0.130704424260806\\
246	-0.107100098599382\\
247	-0.0833999277830344\\
248	-0.0596249622038414\\
249	-0.0357964381957387\\
250	-0.0119357255599439\\
251	0.0119357255599439\\
252	0.0357964381957387\\
253	0.0596249622038414\\
254	0.0833999277830344\\
255	0.107100098599382\\
256	0.130704424260806\\
257	0.154192091887033\\
258	0.177542576530065\\
259	0.200735690208518\\
260	0.223751629329243\\
261	0.246571020285558\\
262	0.269174963029286\\
263	0.291545072438595\\
264	0.31366351731009\\
265	0.335513056831296\\
266	0.357077074397942\\
267	0.378339608672036\\
268	0.399285381781773\\
269	0.419899824599383\\
270	0.440169099038421\\
271	0.460080117344503\\
272	0.479620558361908\\
273	0.498778880785942\\
274	0.517544333422322\\
275	0.535906962497315\\
276	0.553857616074644\\
277	0.57138794565315\\
278	0.588490405031195\\
279	0.605158246539297\\
280	0.621385514749981\\
281	0.637167037787734\\
282	0.652498416369944\\
283	0.667376010714733\\
284	0.681796925459963\\
285	0.695758992743407\\
286	0.709260753593099\\
287	0.722301437785546\\
288	0.734880942321112\\
289	0.746999808678481\\
290	0.7586591989946\\
291	0.769860871326351\\
292	0.780607154137627\\
293	0.790900920160738\\
294	0.800745559765019\\
295	0.810144953973015\\
296	0.819103447246157\\
297	0.827625820166595\\
298	0.83571726212476\\
299	0.843383344126423\\
300	0.850629991812362\\
301	0.857463458791052\\
302	0.863890300363934\\
303	0.86991734772436\\
304	0.875551682700753\\
305	0.88080061310535\\
306	0.885671648746609\\
307	0.890172478153417\\
308	0.894310946052315\\
309	0.898095031636927\\
310	0.901532827654208\\
311	0.904632520335689\\
312	0.907402370189878\\
313	0.909850693667522\\
314	0.911985845710355\\
315	0.913816203184545\\
316	0.915350149198311\\
317	0.916596058301238\\
318	0.917562282552922\\
319	0.918257138453811\\
320	0.918688894721882\\
321	0.918865760898129\\
322	0.91879587676285\\
323	0.918487302542061\\
324	0.917948009879126\\
325	0.91718587355146\\
326	0.916208663903006\\
327	0.915024039969445\\
328	0.913639543267308\\
329	0.912062592221332\\
330	0.910300477200998\\
331	0.908360356137976\\
332	0.90624925069946\\
333	0.90397404298475\\
334	0.901541472723153\\
335	0.898958134940528\\
336	0.896230478071338\\
337	0.893364802489351\\
338	0.890367259429148\\
339	0.887243850276391\\
340	0.884000426200195\\
341	0.880642688105553\\
342	0.877176186882025\\
343	0.87360632392766\\
344	0.869938351925796\\
345	0.866177375856125\\
346	0.862328354218995\\
347	0.858396100456034\\
348	0.854385284547767\\
349	0.850300434772322\\
350	0.846145939610352\\
351	0.841926049778579\\
352	0.8376448803795\\
353	0.833306413153778\\
354	0.82891449882144\\
355	0.824472859501348\\
356	0.819985091196251\\
357	0.81545466633461\\
358	0.810884936357859\\
359	0.806279134344869\\
360	0.801640377665152\\
361	0.796971670651878\\
362	0.792275907289213\\
363	0.787555873905088\\
364	0.782814251865371\\
365	0.778053620261314\\
366	0.773276458587677\\
367	0.768485149404494\\
368	0.763681980979314\\
369	0.758869149905763\\
370	0.754048763694684\\
371	0.749222843334518\\
372	0.744393325818713\\
373	0.739562066636967\\
374	0.734730842227727\\
375	0.729901352391416\\
376	0.725075222660738\\
377	0.720254006628203\\
378	0.715439188228579\\
379	0.710632183975815\\
380	0.705834345153065\\
381	0.701046959955587\\
382	0.696271255585952\\
383	0.69150840030062\\
384	0.686759505408663\\
385	0.682025627221385\\
386	0.67730776895381\\
387	0.672606882577984\\
388	0.667923870627352\\
389	0.663259587954386\\
390	0.658614843439498\\
391	0.653990401654259\\
392	0.649386984477172\\
393	0.644805272663849\\
394	0.640245907372314\\
395	0.63570949164319\\
396	0.631196591836165\\
397	0.626707739023669\\
398	0.6222434303419\\
399	0.617804130300428\\
400	0.613390272051006\\
401	0.609002258616833\\
402	0.604640464082369\\
403	0.600305234745502\\
404	0.595996890231945\\
405	0.591715724573767\\
406	0.587462007251975\\
407	0.58323598420472\\
408	0.579037878801802\\
409	0.57486789278606\\
410	0.570726207182872\\
411	0.566612983178718\\
412	0.562528362968973\\
413	0.558472470576685\\
414	0.554445412642739\\
415	0.550447279187516\\
416	0.546478144346541\\
417	0.542538067079441\\
418	0.538627091853066\\
419	0.534745249301025\\
420	0.530892556857978\\
421	0.527069019371833\\
422	0.52327462969318\\
423	0.519509369242517\\
424	0.515773208556922\\
425	0.512066107816427\\
426	0.508388017350008\\
427	0.50473887812311\\
428	0.501118622206533\\
429	0.497527173227006\\
430	0.493964446801574\\
431	0.490430350953872\\
432	0.486924786515335\\
433	0.483447647510538\\
434	0.479998821527143\\
435	0.47657819007199\\
436	0.473185628912027\\
437	0.469821008403297\\
438	0.466484193805172\\
439	0.463175045582975\\
440	0.459893419698432\\
441	0.456639167887527\\
442	0.453412137928133\\
443	0.450212173895892\\
444	0.447039116409851\\
445	0.443892802867989\\
446	0.440773067672762\\
447	0.437679742447603\\
448	0.434612656243658\\
449	0.431571635738397\\
450	0.428556505425508\\
451	0.425567087796601\\
452	0.422603203515255\\
453	0.419664671583373\\
454	0.416751309500434\\
455	0.413862933415476\\
456	0.410999358272743\\
457	0.408160397950539\\
458	0.405345865394047\\
459	0.402555572741985\\
460	0.399789331447735\\
461	0.39704695239454\\
462	0.394328246006062\\
463	0.39163302235071\\
464	0.388961091242361\\
465	0.38631226233532\\
466	0.383686345215093\\
467	0.381083149484756\\
468	0.378502484847481\\
469	0.37594416118397\\
470	0.373407988627594\\
471	0.370893777634009\\
472	0.368401339048887\\
473	0.36593048417096\\
474	0.363481024812238\\
475	0.361052773355208\\
476	0.358645542806855\\
477	0.35625914684946\\
478	0.353893399889067\\
479	0.351548117101199\\
480	0.349223114473336\\
481	0.346918208846176\\
482	0.344633217951085\\
483	0.342367960446506\\
484	0.340122255951325\\
485	0.337895925076648\\
486	0.335688789455411\\
487	0.333500671769862\\
488	0.331331395777834\\
489	0.329180786336387\\
490	0.327048669424581\\
491	0.32493487216394\\
492	0.322839222838027\\
493	0.32076155091012\\
494	0.318701687039393\\
495	0.316659463096229\\
496	0.314634712175614\\
497	0.312627268610068\\
498	0.310636967980518\\
499	0.308663647126579\\
500	0.306707144156118\\
501	0.306707144156118\\
};
\end{axis}
\end{tikzpicture}%}
    \caption{Elektrisches Feld im Axon bei I = 1mA}
    \label{fig:E1}
\end{figure}

\begin{figure}[h!]
  	\centering
    \scalebox{.6}{% This file was created by matlab2tikz.
% Minimal pgfplots version: 1.3
%
%The latest updates can be retrieved from
%  http://www.mathworks.com/matlabcentral/fileexchange/22022-matlab2tikz
%where you can also make suggestions and rate matlab2tikz.
%
\begin{tikzpicture}

\begin{axis}[%
width=4.520833in,
height=3.565625in,
at={(0.758333in,0.48125in)},
scale only axis,
separate axis lines,
every outer x axis line/.append style={black},
every x tick label/.append style={font=\color{black}},
xmin=1,
xmax=501,
xtick={1,51,101,151,201,251,301,351,401,451,501},
xticklabels={{-25},{-20},{-15},{-10},{-5},{0},{5},{10},{15},{20},{25}},
xlabel={$\text{x distance in }\mu\text{m}$},
every outer y axis line/.append style={black},
every y tick label/.append style={font=\color{black}},
ymin=-0.25,
ymax=0.1,
ylabel={$\text{Activationfunction in V/ }\mu\text{m}^\text{2}$},
title={Activation with I = 1mA}
]
\addplot [color=blue,solid,forget plot]
  table[row sep=crcr]{%
1	0.0195650297046157\\
2	0.0197332085393853\\
3	0.0199030062955074\\
4	0.0200744356554594\\
5	0.0202475092061505\\
6	0.0204222394316389\\
7	0.0205986387072699\\
8	0.0207767192790698\\
9	0.0209564932591277\\
10	0.0211379726064109\\
11	0.0213211691180604\\
12	0.0215060944144696\\
13	0.0216927599202776\\
14	0.0218811768554872\\
15	0.022071356212372\\
16	0.0222633087467727\\
17	0.0224570449518069\\
18	0.0226525750457895\\
19	0.0228499089509171\\
20	0.0230490562715957\\
21	0.0232500262786317\\
22	0.0234528278786783\\
23	0.0236574696039327\\
24	0.0238639595739443\\
25	0.0240723054835357\\
26	0.0242825145702952\\
27	0.0244945935872209\\
28	0.0247085487792731\\
29	0.0249243858512216\\
30	0.0251421099358495\\
31	0.025361725563755\\
32	0.0255832366351072\\
33	0.0258066463727502\\
34	0.026031957303374\\
35	0.0262591712022697\\
36	0.0264882890704143\\
37	0.0267193110834896\\
38	0.0269522365535124\\
39	0.0271870638847815\\
40	0.0274237905319552\\
41	0.0276624129424974\\
42	0.0279029265206176\\
43	0.0281453255649211\\
44	0.0283896032220454\\
45	0.0286357514273305\\
46	0.028883760849574\\
47	0.0291336208293913\\
48	0.0293853193188198\\
49	0.029638842813462\\
50	0.0298941762890692\\
51	0.0301513031288891\\
52	0.0304102050526112\\
53	0.0306708620394502\\
54	0.0309332522515859\\
55	0.0311973519522723\\
56	0.031463135418619\\
57	0.0317305748604113\\
58	0.031999640322411\\
59	0.0322702995939394\\
60	0.0325425181090466\\
61	0.0328162588454361\\
62	0.0330914822219697\\
63	0.0333681459812496\\
64	0.033646205087301\\
65	0.0339256115996278\\
66	0.0342063145515326\\
67	0.0344882598339424\\
68	0.0347713900479718\\
69	0.0350556443853733\\
70	0.0353409584770148\\
71	0.0356272642543232\\
72	0.0359144897952746\\
73	0.0362025591657655\\
74	0.036491392268978\\
75	0.0367809046641909\\
76	0.0370710074049541\\
77	0.0373616068559457\\
78	0.0376526045066328\\
79	0.0379438967865298\\
80	0.0382353748614506\\
81	0.0385269244304709\\
82	0.0388184255204038\\
83	0.0391097522637551\\
84	0.0394007726709944\\
85	0.0396913484097539\\
86	0.0399813345522304\\
87	0.0402705793394631\\
88	0.0405589239228732\\
89	0.0408462020974554\\
90	0.0411322400415415\\
91	0.0414168560318728\\
92	0.0416998601574292\\
93	0.0419810540291721\\
94	0.0422602304725572\\
95	0.0425371732179158\\
96	0.042811656581776\\
97	0.0430834451355722\\
98	0.0433522933686703\\
99	0.0436179453446428\\
100	0.0438801343417339\\
101	0.0441385824942131\\
102	0.0443930004147219\\
103	0.0446430868176861\\
104	0.0448885281249645\\
105	0.045128998070254\\
106	0.0453641572912389\\
107	0.0455936529153433\\
108	0.045817118133229\\
109	0.0460341717708701\\
110	0.0462444178523924\\
111	0.0464474451488783\\
112	0.0466428267296592\\
113	0.0468301195063248\\
114	0.0470088637582577\\
115	0.0471785826757554\\
116	0.0473387818727744\\
117	0.0474889489195718\\
118	0.0476285528533182\\
119	0.0477570436963504\\
120	0.0478738519747779\\
121	0.0479783882274987\\
122	0.0480700425276481\\
123	0.0481481839962328\\
124	0.048212160325356\\
125	0.0482612973067731\\
126	0.0482948983631104\\
127	0.0483122440924077\\
128	0.0483125918174565\\
129	0.0482951751580529\\
130	0.0482592036016527\\
131	0.0482038621107961\\
132	0.0481283107355068\\
133	0.0480316842518036\\
134	0.0479130918318305\\
135	0.0477716167363695\\
136	0.0476063160405715\\
137	0.0474162203971673\\
138	0.0472003338412463\\
139	0.0469576336266542\\
140	0.0466870701327338\\
141	0.0463875667971791\\
142	0.046058020129891\\
143	0.0456972997675109\\
144	0.0453042486164179\\
145	0.0448776830509701\\
146	0.0444163932009189\\
147	0.0439191433233788\\
148	0.0433846722572184\\
149	0.0428116939907852\\
150	0.0421988983177357\\
151	0.0415449516196986\\
152	0.0408484977544532\\
153	0.0401081590826635\\
154	0.0393225376296158\\
155	0.0384902163713008\\
156	0.0376097606967107\\
157	0.0366797200186397\\
158	0.0356986295436457\\
159	0.0346650122352798\\
160	0.0335773809464257\\
161	0.0324342407619582\\
162	0.0312340915275655\\
163	0.0299754306020361\\
164	0.0286567558198669\\
165	0.027276568691903\\
166	0.0258333778262454\\
167	0.0243257026159682\\
168	0.0227520771471035\\
169	0.0211110543851589\\
170	0.019401210630221\\
171	0.0176211502033396\\
172	0.0157695104597622\\
173	0.0138449670213703\\
174	0.0118462393356111\\
175	0.00977209648453936\\
176	0.0076213632766553\\
177	0.00539292662935509\\
178	0.00308574220788671\\
179	0.000698841352786417\\
180	-0.00176866176246904\\
181	-0.00431756268071126\\
182	-0.00694855900889024\\
183	-0.00966224251683911\\
184	-0.012459091029271\\
185	-0.0153394601376533\\
186	-0.0183035747419069\\
187	-0.0213515204283254\\
188	-0.0244832347764401\\
189	-0.0276984985418949\\
190	-0.030996926814808\\
191	-0.0343779601728045\\
192	-0.0378408558461274\\
193	-0.041384678988976\\
194	-0.0450082940680829\\
195	-0.0487103564125846\\
196	-0.0524893040459773\\
197	-0.056343349763921\\
198	-0.0602704736042625\\
199	-0.0642684157288187\\
200	-0.0683346697869069\\
201	-0.0724664768593897\\
202	-0.0766608200166274\\
203	-0.0809144195816458\\
204	-0.0852237292043867\\
205	-0.0895849327314124\\
206	-0.0939939420799618\\
207	-0.0984463960428172\\
208	-0.102937660231106\\
209	-0.107462828112759\\
210	-0.11201672331751\\
211	-0.116593903161188\\
212	-0.12118866357369\\
213	-0.125795045355659\\
214	-0.130406841924469\\
215	-0.135017608496923\\
216	-0.13962067283444\\
217	-0.144209147452301\\
218	-0.148775943447887\\
219	-0.153313785822107\\
220	-0.157815230377523\\
221	-0.162272682106845\\
222	-0.166678415081023\\
223	-0.171024593780444\\
224	-0.175303295785056\\
225	-0.179506535773299\\
226	-0.183626290749928\\
227	-0.187654526363801\\
228	-0.191583224240333\\
229	-0.195404410174049\\
230	-0.199110183060824\\
231	-0.202692744390376\\
232	-0.206144428176103\\
233	-0.209457731097373\\
234	-0.212625342740935\\
235	-0.215640175666465\\
236	-0.218495395212059\\
237	-0.221184448714951\\
238	-0.223701094093087\\
239	-0.226039427437286\\
240	-0.228193909563146\\
241	-0.230159391207252\\
242	-0.231931136784524\\
243	-0.233504846430321\\
244	-0.234876676262274\\
245	-0.236043256614238\\
246	-0.237001708163476\\
247	-0.23774965579193\\
248	-0.238285240081026\\
249	-0.238607126357948\\
250	-0.238714511198879\\
251	-0.238607126357948\\
252	-0.238285240081026\\
253	-0.23774965579193\\
254	-0.237001708163476\\
255	-0.236043256614238\\
256	-0.234876676262274\\
257	-0.233504846430321\\
258	-0.231931136784524\\
259	-0.230159391207252\\
260	-0.228193909563146\\
261	-0.226039427437286\\
262	-0.223701094093087\\
263	-0.221184448714951\\
264	-0.218495395212059\\
265	-0.215640175666465\\
266	-0.212625342740935\\
267	-0.209457731097373\\
268	-0.206144428176103\\
269	-0.202692744390376\\
270	-0.199110183060824\\
271	-0.195404410174049\\
272	-0.191583224240333\\
273	-0.187654526363801\\
274	-0.183626290749928\\
275	-0.179506535773299\\
276	-0.175303295785056\\
277	-0.171024593780444\\
278	-0.166678415081023\\
279	-0.162272682106845\\
280	-0.157815230377523\\
281	-0.153313785822107\\
282	-0.148775943447887\\
283	-0.144209147452301\\
284	-0.13962067283444\\
285	-0.135017608496923\\
286	-0.130406841924469\\
287	-0.125795045355659\\
288	-0.12118866357369\\
289	-0.116593903161188\\
290	-0.11201672331751\\
291	-0.107462828112759\\
292	-0.102937660231106\\
293	-0.0984463960428172\\
294	-0.0939939420799618\\
295	-0.0895849327314124\\
296	-0.0852237292043867\\
297	-0.0809144195816458\\
298	-0.0766608200166274\\
299	-0.0724664768593897\\
300	-0.0683346697869069\\
301	-0.0642684157288187\\
302	-0.0602704736042625\\
303	-0.056343349763921\\
304	-0.0524893040459773\\
305	-0.0487103564125846\\
306	-0.0450082940680829\\
307	-0.041384678988976\\
308	-0.0378408558461274\\
309	-0.0343779601728045\\
310	-0.030996926814808\\
311	-0.0276984985418949\\
312	-0.0244832347764401\\
313	-0.0213515204283254\\
314	-0.0183035747419069\\
315	-0.0153394601376533\\
316	-0.012459091029271\\
317	-0.00966224251683911\\
318	-0.00694855900889024\\
319	-0.00431756268071126\\
320	-0.00176866176246904\\
321	0.000698841352786417\\
322	0.00308574220788671\\
323	0.00539292662935509\\
324	0.0076213632766553\\
325	0.00977209648453936\\
326	0.0118462393356111\\
327	0.0138449670213703\\
328	0.0157695104597622\\
329	0.0176211502033396\\
330	0.019401210630221\\
331	0.0211110543851589\\
332	0.0227520771471035\\
333	0.0243257026159682\\
334	0.0258333778262454\\
335	0.027276568691903\\
336	0.0286567558198669\\
337	0.0299754306020361\\
338	0.0312340915275655\\
339	0.0324342407619582\\
340	0.0335773809464257\\
341	0.0346650122352798\\
342	0.0356986295436457\\
343	0.0366797200186397\\
344	0.0376097606967107\\
345	0.0384902163713008\\
346	0.0393225376296158\\
347	0.0401081590826635\\
348	0.0408484977544532\\
349	0.0415449516196986\\
350	0.0421988983177357\\
351	0.0428116939907852\\
352	0.0433846722572184\\
353	0.0439191433233788\\
354	0.0444163932009189\\
355	0.0448776830509701\\
356	0.0453042486164179\\
357	0.0456972997675109\\
358	0.046058020129891\\
359	0.0463875667971791\\
360	0.0466870701327338\\
361	0.0469576336266542\\
362	0.0472003338412463\\
363	0.0474162203971673\\
364	0.0476063160405715\\
365	0.0477716167363695\\
366	0.0479130918318305\\
367	0.0480316842518036\\
368	0.0481283107355068\\
369	0.0482038621107961\\
370	0.0482592036016527\\
371	0.0482951751580529\\
372	0.0483125918174565\\
373	0.0483122440924077\\
374	0.0482948983631104\\
375	0.0482612973067731\\
376	0.048212160325356\\
377	0.0481481839962328\\
378	0.0480700425276481\\
379	0.0479783882274987\\
380	0.0478738519747779\\
381	0.0477570436963504\\
382	0.0476285528533182\\
383	0.0474889489195718\\
384	0.0473387818727744\\
385	0.0471785826757554\\
386	0.0470088637582577\\
387	0.0468301195063248\\
388	0.0466428267296592\\
389	0.0464474451488783\\
390	0.0462444178523924\\
391	0.0460341717708701\\
392	0.045817118133229\\
393	0.0455936529153433\\
394	0.0453641572912389\\
395	0.045128998070254\\
396	0.0448885281249645\\
397	0.0446430868176861\\
398	0.0443930004147219\\
399	0.0441385824942131\\
400	0.0438801343417339\\
401	0.0436179453446428\\
402	0.0433522933686703\\
403	0.0430834451355722\\
404	0.042811656581776\\
405	0.0425371732179158\\
406	0.0422602304725572\\
407	0.0419810540291721\\
408	0.0416998601574292\\
409	0.0414168560318728\\
410	0.0411322400415415\\
411	0.0408462020974554\\
412	0.0405589239228732\\
413	0.0402705793394631\\
414	0.0399813345522304\\
415	0.0396913484097539\\
416	0.0394007726709944\\
417	0.0391097522637551\\
418	0.0388184255204038\\
419	0.0385269244304709\\
420	0.0382353748614506\\
421	0.0379438967865298\\
422	0.0376526045066328\\
423	0.0373616068559457\\
424	0.0370710074049541\\
425	0.0367809046641909\\
426	0.036491392268978\\
427	0.0362025591657655\\
428	0.0359144897952746\\
429	0.0356272642543232\\
430	0.0353409584770148\\
431	0.0350556443853733\\
432	0.0347713900479718\\
433	0.0344882598339424\\
434	0.0342063145515326\\
435	0.0339256115996278\\
436	0.033646205087301\\
437	0.0333681459812496\\
438	0.0330914822219697\\
439	0.0328162588454361\\
440	0.0325425181090466\\
441	0.0322702995939394\\
442	0.031999640322411\\
443	0.0317305748604113\\
444	0.031463135418619\\
445	0.0311973519522723\\
446	0.0309332522515859\\
447	0.0306708620394502\\
448	0.0304102050526112\\
449	0.0301513031288891\\
450	0.0298941762890692\\
451	0.029638842813462\\
452	0.0293853193188198\\
453	0.0291336208293913\\
454	0.028883760849574\\
455	0.0286357514273305\\
456	0.0283896032220454\\
457	0.0281453255649211\\
458	0.0279029265206176\\
459	0.0276624129424974\\
460	0.0274237905319552\\
461	0.0271870638847815\\
462	0.0269522365535124\\
463	0.0267193110834896\\
464	0.0264882890704143\\
465	0.0262591712022697\\
466	0.026031957303374\\
467	0.0258066463727502\\
468	0.0255832366351072\\
469	0.025361725563755\\
470	0.0251421099358495\\
471	0.0249243858512216\\
472	0.0247085487792731\\
473	0.0244945935872209\\
474	0.0242825145702952\\
475	0.0240723054835357\\
476	0.0238639595739443\\
477	0.0236574696039327\\
478	0.0234528278786783\\
479	0.0232500262786317\\
480	0.0230490562715957\\
481	0.0228499089509171\\
482	0.0226525750457895\\
483	0.0224570449518069\\
484	0.0222633087467727\\
485	0.022071356212372\\
486	0.0218811768554872\\
487	0.0216927599202776\\
488	0.0215060944144696\\
489	0.0213211691180604\\
490	0.0211379726064109\\
491	0.0209564932591277\\
492	0.0207767192790698\\
493	0.0205986387072699\\
494	0.0204222394316389\\
495	0.0202475092061505\\
496	0.0200744356554594\\
497	0.0199030062955074\\
498	0.0197332085393853\\
499	0.0195650297046157\\
500	0.0195650297046157\\
};
\end{axis}
\end{tikzpicture}%}
    \caption{Aktivierungsfunktion im Axon bei I = 1mA}
    \label{fig:A1}
\end{figure}

Die anderen drei Graphen zeigen den Potenzialverlauf (\ref{fig:Phi2}), das elektrische Feld (\ref{fig:E2}) und die Aktivierungsfunktion (\ref{fig:A2}) bei einem Elektrodenstrom von -1mA.

\begin{figure}[h!]
  	\centering
    \scalebox{.6}{\input{img/Phi2.tikz}}
    \caption{Potenzialverlauf im Axon bei I = -1mA}
    \label{fig:Phi2}
\end{figure}

\begin{figure}[h!]
  	\centering
    \scalebox{.6}{\input{img/E2.tikz}}
    \caption{Elektrisches Feld im Axon bei I = -1mA}
    \label{fig:E2}
\end{figure}

\begin{figure}[h!]
  	\centering
    \scalebox{.6}{% This file was created by matlab2tikz.
% Minimal pgfplots version: 1.3
%
%The latest updates can be retrieved from
%  http://www.mathworks.com/matlabcentral/fileexchange/22022-matlab2tikz
%where you can also make suggestions and rate matlab2tikz.
%
\begin{tikzpicture}

\begin{axis}[%
width=4.520833in,
height=3.565625in,
at={(0.758333in,0.48125in)},
scale only axis,
separate axis lines,
every outer x axis line/.append style={black},
every x tick label/.append style={font=\color{black}},
xmin=1,
xmax=501,
xtick={1,51,101,151,201,251,301,351,401,451,501},
xticklabels={{-25},{-20},{-15},{-10},{-5},{0},{5},{10},{15},{20},{25}},
xlabel={$\text{x distance in }\mu\text{m}$},
every outer y axis line/.append style={black},
every y tick label/.append style={font=\color{black}},
ymin=-50,
ymax=250,
ylabel={$\text{Activationfunction in mV/ }\mu\text{m}^\text{2}$},
title={Activation with I = -1mA}
]
\addplot [color=blue,solid,forget plot]
  table[row sep=crcr]{%
1	-19.5650297047905\\
2	-19.733208539219\\
3	-19.9030062954989\\
4	-20.0744356554424\\
5	-20.2475092060922\\
6	-20.4222394317185\\
7	-20.5986387072699\\
8	-20.7767192790925\\
9	-20.9564932591093\\
10	-21.1379726064479\\
11	-21.3211691179822\\
12	-21.506094414508\\
13	-21.69275992037\\
14	-21.8811768552769\\
15	-22.0713562126548\\
16	-22.263308746551\\
17	-22.4570449518069\\
18	-22.6525750458677\\
19	-22.8499089509569\\
20	-23.0490562715204\\
21	-23.2500262785834\\
22	-23.4528278788275\\
23	-23.6574696038588\\
24	-23.8639595738277\\
25	-24.0723054837872\\
26	-24.2825145700408\\
27	-24.4945935874057\\
28	-24.7085487791992\\
29	-24.9243858512273\\
30	-25.1421099357685\\
31	-25.3617255639256\\
32	-25.5832366348841\\
33	-25.8066463729847\\
34	-26.0319573031666\\
35	-26.2591712024005\\
36	-26.488289070403\\
37	-26.7193110834341\\
38	-26.9522365535522\\
39	-27.1870638847758\\
40	-27.4237905319751\\
41	-27.6624129424817\\
42	-27.9029265206191\\
43	-28.1453255649467\\
44	-28.3896032220582\\
45	-28.6357514272822\\
46	-28.8837608495669\\
47	-29.1336208294524\\
48	-29.3853193186806\\
49	-29.6388428136197\\
50	-29.8941762888717\\
51	-30.1513031290597\\
52	-30.4102050526126\\
53	-30.6708620393692\\
54	-30.9332522516343\\
55	-31.1973519523235\\
56	-31.4631354185622\\
57	-31.7305748603758\\
58	-31.999640322465\\
59	-32.2702995939835\\
60	-32.5425181088576\\
61	-32.8162588455598\\
62	-33.0914822219711\\
63	-33.3681459813306\\
64	-33.6462050872797\\
65	-33.9256115994431\\
66	-34.2063145517386\\
67	-34.488259833779\\
68	-34.7713900480812\\
69	-35.0556443852838\\
70	-35.3409584771725\\
71	-35.6272642542535\\
72	-35.914489795141\\
73	-36.2025591659403\\
74	-36.4913922689084\\
75	-36.7809046641923\\
76	-37.0710074050294\\
77	-37.3616068558476\\
78	-37.6526045065475\\
79	-37.9438967867827\\
80	-38.2353748611422\\
81	-38.5269244306983\\
82	-38.8184255203669\\
83	-39.109752263721\\
84	-39.4007726710697\\
85	-39.6913484095421\\
86	-39.9813345524308\\
87	-40.2705793394489\\
88	-40.5589239227993\\
89	-40.8462020974184\\
90	-41.1322400417703\\
91	-41.4168560317194\\
92	-41.6998601573141\\
93	-41.9810540293838\\
94	-42.2602304723114\\
95	-42.5371732180793\\
96	-42.8116565817618\\
97	-43.0834451355622\\
98	-43.3522933686618\\
99	-43.6179453447039\\
100	-43.8801343416344\\
101	-44.1385824942699\\
102	-44.3930004148569\\
103	-44.6430868174502\\
104	-44.8885281250114\\
105	-45.1289980703223\\
106	-45.3641572912602\\
107	-45.5936529153405\\
108	-45.8171181331636\\
109	-46.0341717709525\\
110	-46.2444178523583\\
111	-46.4474451488058\\
112	-46.6428267296578\\
113	-46.8301195063759\\
114	-47.0088637583103\\
115	-47.1785826755877\\
116	-47.3387818728952\\
117	-47.4889489196357\\
118	-47.6285528533481\\
119	-47.7570436962196\\
120	-47.8738519746912\\
121	-47.9783882276024\\
122	-48.0700425277973\\
123	-48.1481839960907\\
124	-48.2121603252381\\
125	-48.261297306999\\
126	-48.2948983630194\\
127	-48.312244092449\\
128	-48.3125918173755\\
129	-48.2951751580913\\
130	-48.2592036016285\\
131	-48.2038621108586\\
132	-48.1283107354102\\
133	-48.0316842518732\\
134	-47.9130918318333\\
135	-47.7716167362814\\
136	-47.6063160405829\\
137	-47.416220397281\\
138	-47.2003338411014\\
139	-46.9576336266982\\
140	-46.6870701329753\\
141	-46.3875667970569\\
142	-46.058020129567\\
143	-45.697299767744\\
144	-45.3042486165941\\
145	-44.8776830506176\\
146	-44.4163932013907\\
147	-43.9191433230007\\
148	-43.3846722571616\\
149	-42.8116939910978\\
150	-42.1988983176561\\
151	-41.5449516196531\\
152	-40.8484977542685\\
153	-40.1081590829563\\
154	-39.322537629414\\
155	-38.4902163714287\\
156	-37.6097606964322\\
157	-36.6797200189467\\
158	-35.698629543549\\
159	-34.6650122352003\\
160	-33.5773809463717\\
161	-32.4342407620861\\
162	-31.2340915275854\\
163	-29.9754306022805\\
164	-28.6567558196111\\
165	-27.276568691741\\
166	-25.8333778263477\\
167	-24.325702616261\\
168	-22.7520771466516\\
169	-21.1110543856194\\
170	-19.4012106298032\\
171	-17.6211502035585\\
172	-15.7695104597224\\
173	-13.844967021214\\
174	-11.8462393358641\\
175	-9.77209648444841\\
176	-7.62136327648477\\
177	-5.39292662979278\\
178	-3.08574220762239\\
179	-0.698841352641466\\
180	1.76866176225303\\
181	4.3175626808079\\
182	6.94855900910625\\
183	9.66224251642416\\
184	12.4590910294501\\
185	15.3394601376931\\
186	18.3035747417307\\
187	21.3515204286523\\
188	24.4832347761985\\
189	27.6984985419403\\
190	30.9969268149871\\
191	34.3779601724236\\
192	37.8408558462979\\
193	41.3846789891977\\
194	45.008294067884\\
195	48.7103564126301\\
196	52.4893040459574\\
197	56.3433497638471\\
198	60.2704736043961\\
199	64.2684157286567\\
200	68.3346697871457\\
201	72.4664768593357\\
202	76.6608200163318\\
203	80.9144195820409\\
204	85.2237292041536\\
205	89.5849327313044\\
206	93.9939420801238\\
207	98.446396043073\\
208	102.9376602306\\
209	107.462828113057\\
210	112.016723317356\\
211	116.593903161265\\
212	121.188663573776\\
213	125.795045355699\\
214	130.406841924196\\
215	135.017608497219\\
216	139.62067283428\\
217	144.20914745242\\
218	148.775943447617\\
219	153.31378582232\\
220	157.815230377673\\
221	162.272682106777\\
222	166.678415080605\\
223	171.024593780749\\
224	175.303295785125\\
225	179.50653577318\\
226	183.62629075018\\
227	187.654526363622\\
228	191.583224240094\\
229	195.404410174524\\
230	199.110183060475\\
231	202.692744390515\\
232	206.144428176049\\
233	209.457731097427\\
234	212.625342740648\\
235	215.640175666704\\
236	218.495395212085\\
237	221.184448715212\\
238	223.701094092758\\
239	226.039427437354\\
240	228.193909562833\\
241	230.159391207781\\
242	231.93113678426\\
243	233.504846430151\\
244	234.876676262502\\
245	236.043256614357\\
246	237.00170816337\\
247	237.749655791777\\
248	238.285240081313\\
249	238.607126357601\\
250	238.714511199214\\
251	238.607126357601\\
252	238.285240081313\\
253	237.749655791777\\
254	237.00170816337\\
255	236.043256614357\\
256	234.876676262502\\
257	233.504846430151\\
258	231.93113678426\\
259	230.159391207781\\
260	228.193909562833\\
261	226.039427437354\\
262	223.701094092758\\
263	221.184448715212\\
264	218.495395212085\\
265	215.640175666704\\
266	212.625342740648\\
267	209.457731097427\\
268	206.144428176049\\
269	202.692744390515\\
270	199.110183060475\\
271	195.404410174524\\
272	191.583224240094\\
273	187.654526363622\\
274	183.62629075018\\
275	179.50653577318\\
276	175.303295785125\\
277	171.024593780749\\
278	166.678415080605\\
279	162.272682106777\\
280	157.815230377673\\
281	153.31378582232\\
282	148.775943447617\\
283	144.20914745242\\
284	139.62067283428\\
285	135.017608497219\\
286	130.406841924196\\
287	125.795045355699\\
288	121.188663573776\\
289	116.593903161265\\
290	112.016723317356\\
291	107.462828113057\\
292	102.9376602306\\
293	98.446396043073\\
294	93.9939420801238\\
295	89.5849327313044\\
296	85.2237292041536\\
297	80.9144195820409\\
298	76.6608200163318\\
299	72.4664768593357\\
300	68.3346697871457\\
301	64.2684157286567\\
302	60.2704736043961\\
303	56.3433497638471\\
304	52.4893040459574\\
305	48.7103564126301\\
306	45.008294067884\\
307	41.3846789891977\\
308	37.8408558462979\\
309	34.3779601724236\\
310	30.9969268149871\\
311	27.6984985419403\\
312	24.4832347761985\\
313	21.3515204286523\\
314	18.3035747417307\\
315	15.3394601376931\\
316	12.4590910294501\\
317	9.66224251642416\\
318	6.94855900910625\\
319	4.3175626808079\\
320	1.76866176225303\\
321	-0.698841352641466\\
322	-3.08574220762239\\
323	-5.39292662979278\\
324	-7.62136327648477\\
325	-9.77209648444841\\
326	-11.8462393358641\\
327	-13.844967021214\\
328	-15.7695104597224\\
329	-17.6211502035585\\
330	-19.4012106298032\\
331	-21.1110543856194\\
332	-22.7520771466516\\
333	-24.325702616261\\
334	-25.8333778263477\\
335	-27.276568691741\\
336	-28.6567558196111\\
337	-29.9754306022805\\
338	-31.2340915275854\\
339	-32.4342407620861\\
340	-33.5773809463717\\
341	-34.6650122352003\\
342	-35.698629543549\\
343	-36.6797200189467\\
344	-37.6097606964322\\
345	-38.4902163714287\\
346	-39.322537629414\\
347	-40.1081590829563\\
348	-40.8484977542685\\
349	-41.5449516196531\\
350	-42.1988983176561\\
351	-42.8116939910978\\
352	-43.3846722571616\\
353	-43.9191433230007\\
354	-44.4163932013907\\
355	-44.8776830506176\\
356	-45.3042486165941\\
357	-45.697299767744\\
358	-46.058020129567\\
359	-46.3875667970569\\
360	-46.6870701329753\\
361	-46.9576336266982\\
362	-47.2003338411014\\
363	-47.416220397281\\
364	-47.6063160405829\\
365	-47.7716167362814\\
366	-47.9130918318333\\
367	-48.0316842518732\\
368	-48.1283107354102\\
369	-48.2038621108586\\
370	-48.2592036016285\\
371	-48.2951751580913\\
372	-48.3125918173755\\
373	-48.312244092449\\
374	-48.2948983630194\\
375	-48.261297306999\\
376	-48.2121603252381\\
377	-48.1481839960907\\
378	-48.0700425277973\\
379	-47.9783882276024\\
380	-47.8738519746912\\
381	-47.7570436962196\\
382	-47.6285528533481\\
383	-47.4889489196357\\
384	-47.3387818728952\\
385	-47.1785826755877\\
386	-47.0088637583103\\
387	-46.8301195063759\\
388	-46.6428267296578\\
389	-46.4474451488058\\
390	-46.2444178523583\\
391	-46.0341717709525\\
392	-45.8171181331636\\
393	-45.5936529153405\\
394	-45.3641572912602\\
395	-45.1289980703223\\
396	-44.8885281250114\\
397	-44.6430868174502\\
398	-44.3930004148569\\
399	-44.1385824942699\\
400	-43.8801343416344\\
401	-43.6179453447039\\
402	-43.3522933686618\\
403	-43.0834451355622\\
404	-42.8116565817618\\
405	-42.5371732180793\\
406	-42.2602304723114\\
407	-41.9810540293838\\
408	-41.6998601573141\\
409	-41.4168560317194\\
410	-41.1322400417703\\
411	-40.8462020974184\\
412	-40.5589239227993\\
413	-40.2705793394489\\
414	-39.9813345524308\\
415	-39.6913484095421\\
416	-39.4007726710697\\
417	-39.109752263721\\
418	-38.8184255203669\\
419	-38.5269244306983\\
420	-38.2353748611422\\
421	-37.9438967867827\\
422	-37.6526045065475\\
423	-37.3616068558476\\
424	-37.0710074050294\\
425	-36.7809046641923\\
426	-36.4913922689084\\
427	-36.2025591659403\\
428	-35.914489795141\\
429	-35.6272642542535\\
430	-35.3409584771725\\
431	-35.0556443852838\\
432	-34.7713900480812\\
433	-34.488259833779\\
434	-34.2063145517386\\
435	-33.9256115994431\\
436	-33.6462050872797\\
437	-33.3681459813306\\
438	-33.0914822219711\\
439	-32.8162588455598\\
440	-32.5425181088576\\
441	-32.2702995939835\\
442	-31.999640322465\\
443	-31.7305748603758\\
444	-31.4631354185622\\
445	-31.1973519523235\\
446	-30.9332522516343\\
447	-30.6708620393692\\
448	-30.4102050526126\\
449	-30.1513031290597\\
450	-29.8941762888717\\
451	-29.6388428136197\\
452	-29.3853193186806\\
453	-29.1336208294524\\
454	-28.8837608495669\\
455	-28.6357514272822\\
456	-28.3896032220582\\
457	-28.1453255649467\\
458	-27.9029265206191\\
459	-27.6624129424817\\
460	-27.4237905319751\\
461	-27.1870638847758\\
462	-26.9522365535522\\
463	-26.7193110834341\\
464	-26.488289070403\\
465	-26.2591712024005\\
466	-26.0319573031666\\
467	-25.8066463729847\\
468	-25.5832366348841\\
469	-25.3617255639256\\
470	-25.1421099357685\\
471	-24.9243858512273\\
472	-24.7085487791992\\
473	-24.4945935874057\\
474	-24.2825145700408\\
475	-24.0723054837872\\
476	-23.8639595738277\\
477	-23.6574696038588\\
478	-23.4528278788275\\
479	-23.2500262785834\\
480	-23.0490562715204\\
481	-22.8499089509569\\
482	-22.6525750458677\\
483	-22.4570449518069\\
484	-22.263308746551\\
485	-22.0713562126548\\
486	-21.8811768552769\\
487	-21.69275992037\\
488	-21.506094414508\\
489	-21.3211691179822\\
490	-21.1379726064479\\
491	-20.9564932591093\\
492	-20.7767192790925\\
493	-20.5986387072699\\
494	-20.4222394317185\\
495	-20.2475092060922\\
496	-20.0744356554424\\
497	-19.9030062954989\\
498	-19.733208539219\\
499	-19.5650297047905\\
500	-19.5650297047905\\
};
\end{axis}
\end{tikzpicture}%}
    \caption{Aktivierungsfunktion im Axon bei I = -1mA}
    \label{fig:A2}
\end{figure}

\end{enumerate}

\section{Create a Neuron Model}

\begin{enumerate}
\item Einphasiger Strompuls mit Amplitude -0.25mA
\begin{figure}[h!]
	\centering
	\includegraphics[width=0.5\textwidth]{img/mono_neg_025_1.png}
	\caption{Membranspannung mit I = -0.25mA (einphasig)}
	\label{fig:mono_neg_025_1}
\end{figure}

\item Einphasiger Strompuls mit Amplitude -1mA
\begin{figure}[h!]
	\centering
	\includegraphics[width=0.5\textwidth]{img/mono_neg_1_1.png}
	\caption{Membranspannung mit I = -1mA (einphasig)}
	\label{fig:mono_neg_1_1}
\end{figure}

\item Bi-phasiger Strompuls mit Amplitude 0.5mA
\begin{figure}[h!]
	\centering
	\includegraphics[width=0.5\textwidth]{img/bi_05_1.png}
	\caption{Membranspannung mit I = 0.5mA (bi-phasig)}
	\label{fig:bi_05_1}
\end{figure}

\item Bi-phasiger Strompuls mit Amplitude 2mA
\begin{figure}[h!]
	\centering
	\includegraphics[width=0.5\textwidth]{img/bi_2_1.png}
	\caption{Membranspannung mit I = 2mA (bi-phasig)}
	\label{fig:bi_2_1}
\end{figure}

\item Einphasiger Strompuls mit Amplitude 0.25mA
\begin{figure}[h!]
	\centering
	\includegraphics[width=0.5\textwidth]{img/mono_pos_025_1.png}
	\caption{Membranspannung mit I = 0.25mA (einphasig)}
	\label{fig:mono_pos_025_1}
\end{figure}

\item Einphasiger Strompuls mit Amplitude 5mA
\begin{figure}[h!]
	\centering
	\includegraphics[width=0.5\textwidth]{img/mono_pos_5_1.png}
	\caption{Membranspannung mit I = 5mA (einphasig)}
	\label{fig:mono_pos_5_1}
\end{figure}

\end{enumerate}

\end{document}


