\documentclass[conference]{IEEEtran}

% Packages
\usepackage[utf8]{inputenc}
\usepackage[english]{babel}
\usepackage{amsmath}
\usepackage{amsfonts}
\usepackage{amssymb}
\usepackage{amsthm}
\usepackage{pdfpages}
\usepackage{graphicx}
\usepackage{epstopdf}
\usepackage{listings}
\usepackage{cite}
\usepackage{enumerate}
\usepackage{scientific}
\usepackage[colorlinks=false]{hyperref}
\usepackage{bookmark}

\usepackage[]{mcode}	%Matlab Code
\usepackage{tikz,pgfplots}	%Tikz

% Bookmark Setup
\bookmarksetup{numbered}

% PDF Setup
\hypersetup{pdftitle={Homework 4}, pdfsubject={Documentation of 4th Homework}, pdfauthor={Stefan Röhrl}, pdfkeywords={Neuroprothetik Exercise}, pdfcreator={LaTeX}, hidelinks}


\begin{document}
%
% cite all references
%\nocite{*}
%
% paper title
% can use linebreaks \\ within to get better formatting as desired
\title{Homework 4\\ Hodgkin \& Huxley Model}

\author{\IEEEauthorblockN{Stefan Röhrl}
\IEEEauthorblockA{Technische Universität München, Arcisstraße 21, Munich, Germany\\
Email: stefan.roehrl@tum.de}}

% use for special paper notices
%\IEEEspecialpapernotice{(Invited Paper)}

% make the title area
\maketitle

\IEEEpeerreviewmaketitle

\section{Time constants and steady state values}
Wie in der Vorlesung gezeigt wurde gelten für die Gating Variablen folgende Differentialgleichung mit $x \in \{m,n,h\}$
\begin{equation}
	\dot{x} = -(\alpha_x + \beta_x) \cdot \left(x - \frac{\alpha_x}{\alpha_x + \beta_x}\right)
	\label{StandardDGL}
\end{equation}
Möchte man Gleichung \eqref{StandardDGL} nun auf Form aus der Angabe bringen, ergeben sich für $\tau_x$ und $x_{\infty}$ folgende Gleichungen:
\begin{align}
	\tau_x & = \frac{1}{\alpha_x + \beta_x}\\
	x_{\infty} & = \frac{\alpha_x}{\alpha_x + \beta_x}
\end{align}



\section{Hodgkin \& Huxley Neuron Model}
\begin{enumerate}
\item bla
\item bla
\end{enumerate}



\end{document}


