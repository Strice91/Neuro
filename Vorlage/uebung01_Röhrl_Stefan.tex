\documentclass[a4paper, 12pt]{report}

%%% Language
\usepackage[english,ngerman]{babel}

%%% Encoding
\usepackage[utf8]{inputenc} % UTF8

%%% Packages (add new as needed)
\usepackage[margin=3cm]{geometry}
\usepackage{graphicx}
\usepackage{amsmath} % nice equations
\usepackage{url} % URLs
\usepackage{natbib} % author-year bibliography style
\usepackage{baithesis} % BAI thesis
% \usepackage{hyperref} % PDF links
% \usepackage{subfig} % Subfigures (a), (b), etc
% \usepackage{nomencl} % Nomenclature
\usepackage{epstopdf}

%%% Meta data
\author{Stefan Röhrl}
\title{Grafiken der Übung zur Vorlesung Neuroporthetik}
%\degree{Masterarbeit}
\logo{bai-tum}

\begin{document}

% Kopf- und Fußzeile
\pagestyle{fancy}
\fancyhf{} % alle Kopf- und Fusszeilenfelder bereinigen
\renewcommand{\headrulewidth}{0pt} % duenne obere Trennlinie
\renewcommand{\footrulewidth}{0pt} % keine untere Trennlinie
\fancyfoot[L]{Seite~\thepage\ \hfill Stefan Röhrl, SS\,2015} 
\fancyhead[C]{\textit{\nouppercase{\rightmark \includegraphics[width=\textwidth]{bai-tum}}}} % section

\setlength{\headheight}{35pt} % Schiebt Überschrift aus Logo, nur bei der ersten Seite nötig


\section*{Neuroprothetik -- Übung 1}
%\subsection*{Grundlagen der digitalen Signalverarbeitung -- Bericht}
\vspace{.3cm}








% \setlength{\headheight}{10pt} % evtl auf folgenden Seiten zurücksetzen
\end{document}
